\section{Auswertung}
\label{sec:Auswertung}

\subsection{Fehlerrechnung}
\label{sec:Fehlerrechnung}
Für die Fehlerrechnung werden folgende Formeln aus der Vorlesung verwendet.
für den Mittelwert gilt
\begin{equation}
    \overline{x}=\frac{1}{N}\sum_{i=1}^N x_i ß\; \;\text{mit der Anzahl N und den Messwerten x} 
    \label{eqn:Mittelwert}
\end{equation}
Der Fehler für den Mittelwert lässt sich gemäß
\begin{equation}
    \increment \overline{x}=\frac{1}{\sqrt{N}}\sqrt{\frac{1}{N-1}\sum_{i=1}^N(x_i-\overline{x})^2}
    \label{eqn:FehlerMittelwert}
\end{equation}
berechnen.
Wenn im weiteren Verlauf der Berechnung mit der fehlerhaften Größe gerechnet wird, kann der Fehler der folgenden Größe
mittels Gaußscher Fehlerfortpflanzung berechnet werden. Die Formel hierfür ist
\begin{equation}
    \increment f= \sqrt{\sum_{i=1}^N\left(\frac{\partial f}{\partial x_i}\right)^2\cdot(\increment x_i)^2}.
    \label{eqn:GaussMittelwert}
\end{equation}





\subsection{Brechungsindex von Glas}
\label{sec:AuswGlas}
Die Anzahl der Peaks und die aus \autoref{eqn:Brechi} resultierenden Brechindizes sind in \autoref{tab:Glasn} dargestellt.
\begin{table}
    \centering
    \caption{Anzahl der Peaks und der daraus resultierende Brechungsindex}
    \begin{tabular}{c c c}
        \toprule
        Messung & Anzahl der Minima & Brechungsindex $n$ \\
        \midrule
             1 & 34 & 1.546\\
             2 & 34 & 1.546\\
             3 & 35 & 1.571\\
             4 & 35 & 1.571\\
             5 & 34 & 1.546\\
             6 & 34 & 1.546\\
             7 & 36 & 1.597\\
             8 & 34 & 1.546\\
             9 & 35 & 1.571\\
            10 & 33 & 1.522\\
            \bottomrule
    \end{tabular}
    \label{tab:Glasn}
\end{table}
Der daraus resultierende Mittelwert für den Brechungsindex $n$ von Glas beträgt $n=1.556$.
