\section{Auswertung}
\label{sec:Auswertung}

\subsection{Fehlerrechnung}
\label{sec:Fehlerrechnung}
Für die Fehlerrechnung werden folgende Formeln aus der Vorlesung verwendet.
für den Mittelwert gilt
\begin{equation}
    \overline{x}=\frac{1}{N}\sum_{i=1}^N x_i ß\; \;\text{mit der Anzahl N und den Messwerten x} 
    \label{eqn:Mittelwert}
\end{equation}
Der Fehler für den Mittelwert lässt sich gemäß
\begin{equation}
    \increment \overline{x}=\frac{1}{\sqrt{N}}\sqrt{\frac{1}{N-1}\sum_{i=1}^N(x_i-\overline{x})^2}
    \label{eqn:FehlerMittelwert}
\end{equation}
berechnen.
Wenn im weiteren Verlauf der Berechnung mit der fehlerhaften Größe gerechnet wird, kann der Fehler der folgenden Größe
mittels Gaußscher Fehlerfortpflanzung berechnet werden. Die Formel hierfür ist
\begin{equation}
    \increment f= \sqrt{\sum_{i=1}^N\left(\frac{\partial f}{\partial x_i}\right)^2\cdot(\increment x_i)^2}.
    \label{eqn:GaussMittelwert}
\end{equation}

\subsection{Kontrast}

\begin{sidewaystable}
\begin{tabular}{c c c c c c c c c c}
    \toprule
        Winkel/°&\multicolumn{3}{c}{Minima/V}&Mittelwert der Minima&\multicolumn{3}{c}{Maxima/V}&Mittelwert der Maxima&Kontrast \\
    \midrule
      0& 1.55 & 1.48 & 1.48 & 1.50 & 1.83 & 1.78 & 1.78 & 1.80 & 0.089 \\
      5& 1.26 & 1.26 & 1.27 & 1.26 & 1.53 & 1.58 & 1.59 & 1.57 & 0.107 \\
     10& 1.04 & 1.01 & 0.98 & 1.01 & 1.56 & 1.55 & 1.55 & 1.55 & 0.212 \\
     15& 0.67 & 0.68 & 0.68 & 0.68 & 1.56 & 1.56 & 1.53 & 1.55 & 0.392 \\
     20& 0.47 & 0.48 & 0.49 & 0.48 & 1.48 & 1.49 & 1.45 & 1.47 & 0.509 \\
     25& 0.30 & 0.31 & 0.32 & 0.31 & 1.46 & 1.47 & 1.41 & 1.45 & 0.647 \\
     30& 0.21 & 0.19 & 0.18 & 0.19 & 1.30 & 1.34 & 1.39 & 1.34 & 0.748 \\
     35& 0.11 & 0.10 & 0.10 & 0.10 & 1.40 & 1.41 & 1.42 & 1.41 & 0.863 \\
     40& 0.07 & 0.07 & 0.09 & 0.08 & 1.47 & 1.47 & 1.44 & 1.46 & 0.900 \\
     45& 0.05 & 0.05 & 0.07 & 0.06 & 1.53 & 1.52 & 1.52 & 1.52 & 0.928 \\
     50& 0.05 & 0.06 & 0.04 & 0.05 & 1.71 & 1.69 & 1.71 & 1.70 & 0.943 \\
     55& 0.07 & 0.07 & 0.08 & 0.07 & 1.81 & 1.81 & 1.81 & 1.81 & 0.922 \\
     60& 0.15 & 0.13 & 0.15 & 0.14 & 1.94 & 1.91 & 1.92 & 1.92 & 0.861 \\
     65& 0.21 & 0.21 & 0.21 & 0.21 & 2.07 & 2.03 & 2.05 & 2.05 & 0.814 \\
     70& 0.40 & 0.40 & 0.43 & 0.41 & 2.33 & 2.33 & 2.39 & 2.35 & 0.703 \\
     75& 0.69 & 0.62 & 0.62 & 0.64 & 2.20 & 2.26 & 2.26 & 2.24 & 0.554 \\
     80& 1.01 & 0.97 & 0.96 & 0.98 & 2.10 & 2.20 & 2.21 & 2.17 & 0.378 \\
     85& 1.34 & 1.26 & 1.26 & 1.29 & 1.97 & 2.15 & 2.15 & 2.09 & 0.238 \\
     90& 1.78 & 1.71 & 1.75 & 1.75 & 2.06 & 2.07 & 2.15 & 2.09 & 0.090 \\
     95& 1.91 & 1.91 & 1.92 & 1.91 & 2.44 & 2.53 & 2.53 & 2.50 & 0.133 \\
    100& 1.62 & 1.46 & 1.45 & 1.51 & 3.14 & 3.12 & 3.14 & 3.13 & 0.350 \\
    105& 1.19 & 1.15 & 1.16 & 1.17 & 3.66 & 3.68 & 3.65 & 3.66 & 0.517 \\
    110& 1.01 & 0.89 & 0.88 & 0.93 & 4.39 & 4.43 & 4.41 & 4.41 & 0.653 \\
    115& 0.60 & 0.60 & 0.60 & 0.60 & 4.86 & 4.85 & 4.85 & 4.85 & 0.780 \\
    120& 0.41 & 0.40 & 0.40 & 0.40 & 5.45 & 5.44 & 5.45 & 5.45 & 0.862 \\
    125& 0.23 & 0.21 & 0.22 & 0.22 & 5.85 & 5.82 & 5.87 & 5.85 & 0.927 \\
    130& 0.15 & 0.15 & 0.15 & 0.15 & 5.98 & 6.01 & 6.03 & 6.01 & 0.951 \\
    135& 0.15 & 0.14 & 0.15 & 0.15 & 5.90 & 5.90 & 5.99 & 5.93 & 0.952 \\
    140& 0.22 & 0.23 & 0.22 & 0.22 & 5.94 & 5.94 & 5.94 & 5.94 & 0.928 \\
    145& 0.63 & 0.35 & 0.32 & 0.43 & 5.40 & 5.31 & 5.31 & 5.34 & 0.850 \\
    150& 0.55 & 0.51 & 0.52 & 0.52 & 5.11 & 5.09 & 5.10 & 5.10 & 0.813 \\
    155& 0.79 & 0.77 & 0.79 & 0.78 & 4.30 & 4.48 & 4.46 & 4.41 & 0.699 \\
    160& 0.97 & 0.96 & 0.96 & 0.96 & 3.98 & 3.95 & 3.96 & 3.96 & 0.609 \\
    165& 1.13 & 1.14 & 1.13 & 1.13 & 3.15 & 3.19 & 3.14 & 3.16 & 0.472 \\
    170& 1.33 & 1.33 & 1.34 & 1.33 & 2.78 & 2.78 & 2.78 & 2.78 & 0.352 \\
    175& 1.48 & 1.47 & 1.47 & 1.47 & 2.27 & 2.26 & 2.25 & 2.26 & 0.211 \\
    180& 1.51 & 1.50 & 1.50 & 1.50 & 1.69 & 1.69 & 1.68 & 1.69 & 0.057 \\
    \bottomrule
    \end{tabular}
    \label{tab:kontrast}
\end{sidewaystable}

\begin{figure}
    \centering
    \includegraphics[width=0.8\textwidth]{build/kontrast.pdf}
    \caption{Kontrast in Abhängigkeit vom Winkel des Polarisationsfilters.}
    \label{fig:kontrast}
\end{figure}

\subsection{Brechungsindex von Glas}
\label{sec:AuswGlas}
Die Anzahl der Peaks und die aus \autoref{eqn:Brechi} resultierenden Brechindizes sind in \autoref{tab:Glasn} dargestellt.
\begin{table}
    \centering
    \caption{Anzahl der Peaks und der daraus resultierende Brechungsindex.}
    \begin{tabular}{c c c}
        \toprule
        Messung & Anzahl der Minima & Brechungsindex $n$ \\
        \midrule
             1 & 34 & 1.546\\
             2 & 34 & 1.546\\
             3 & 35 & 1.571\\
             4 & 35 & 1.571\\
             5 & 34 & 1.546\\
             6 & 34 & 1.546\\
             7 & 36 & 1.597\\
             8 & 34 & 1.546\\
             9 & 35 & 1.571\\
            10 & 33 & 1.522\\
            \bottomrule
    \end{tabular}
    \label{tab:Glasn}
\end{table}
Der daraus resultierende Mittelwert für den Brechungsindex $n$ von Glas beträgt $n=1.556$.


\subsection{Brechungsindex von Luft}
\label{sec:AusLuft}
In diesem Teil der Auswertung wird der Brechungsindex von Luft ermittelt. Die Daten der Messreihen sind in \autoref{tab:Brechung} niedergeschrieben und in \autoref{fig:allefüreinen} graphisch aufgetragen.
\begin{table}
    \centering
    \caption{Messwerte zur Messung des Brechungsindex von Luft.}
\begin{tabular}{c c c c c c c c }
    \toprule
    Druck $p \mathrm{/} \unit{\milli \bar}$ & Messung 1& Messung 2& Messung 3& Messung 4& Messung 5& Mittelwert \\
    \midrule
                           50& 3& 2& 2& 2& 2& 2.2 \\
                          100& 5& 4& 4& 4& 4& 4.2 \\
                          150& 7& 6& 6& 6& 6& 6.2 \\
                          200& 9& 8& 8& 8& 8& 8.2 \\
                    250& 11& 10& 10& 10& 10& 10.2 \\
                    300& 13& 12& 12& 12& 12& 12.2 \\
                    350& 15& 14& 14& 15& 14& 14.4 \\
                    400& 17& 16& 17& 17& 17& 16.8 \\
                      450& 19& 19& 19& 19& 19& 19 \\
                    500& 22& 21& 21& 21& 21& 21.2 \\
                    550& 24& 23& 23& 23& 23& 23.2 \\
                    600& 26& 25& 25& 25& 25& 25.2&  \\
                    650& 28& 27& 27& 27& 27& 27.2 \\
                    700& 30& 29& 29& 29& 29& 29.2 \\
                    750& 32& 31& 31& 31& 31& 31.2 \\
                    800& 35& 33& 33& 34& 34& 33.8 \\
                    850& 37& 36& 37& 36& 36& 36.4 \\
                    900& 39& 38& 39& 38& 38& 38.4 \\
                    950& 41& 40& 41& 40& 40& 40.4 \\
                   1000& 43& 42& 43& 42& 42& 42.4 \\
    \bottomrule
    \end{tabular}
    \label{tab:Brechung}
\end{table}

\begin{figure}
    \centering
    \includegraphics[height = 6cm]{build/luft.pdf}
    \caption{Messwerte der fünf Messungen zur Bestimmung von $n$.}
    \label{fig:allefüreinen}
\end{figure}
Um aus den Messwerten ein Brechungsindex ermitteln zu können, wird der Durchschnitt der fünf Messungen für den jeweiligen Druck ermittelt und aufgetragen. Die anschließend durchgeführte lineare Regression nach \autoref{eq:nluft2} mit $T=293.75\,\unit{\kelvin}$
liefert die in blau eingezeichnete Ausgleichsgerade mit einem Wert von $M=0.0780\pm 0.0005$.
Dieser Plot ist in \autoref{fig:einerfüralle} dargestellt.
\begin{figure}
    \centering
    \includegraphics[height = 6cm]{build/luftavg.pdf}
    \caption{Durchschnitt der fünf Messungen und lineare Regression.}
    \label{fig:einerfüralle}
\end{figure}
