\section{Durchführung}
\label{sec:Durchführung}
Der Versuch beginnt damit, die einzelnen Bauteile genau zu justieren. Hier wird auf diesen Prozess nicht genauer eingegangen, dieser ist in der Versuchsanleitung
\cite{ap64} beschrieben. \\
\\
Zuerst wird der Kontrast des Interferometers in Abhängigkeit von der Polarisationsrichtung $\phi$ des Laserstrahls gemessen. Dazu werden das Interferenzmaximum und -minimum
mittels einer Photodiode gemessen. Das Interferenzmaximum und -minimum kann durch den Doppelglashalter eingestellt werden.
Die Diodenspannung wird in Abhängigkeit von der Orientierung des Polarisationsfilters vor dem ersten PBSC gemessen. Dazu wird je im Abstand von 5° eine Messung aufgenommen.
\\
\\
Nun werden die Brechungsindizes von Glas und Luft vermessen. Dazu wird der Polarisator so eingestellt, dass der Kontrast maximal ist. 
Um den Brechungsindex von Glas zu bestimmen, wird die Anzahl der Interferenzmaxima in Abhängigkeit vom Drehwinkel 
$\theta$ der Glasplatten aus dem Doppelglashalter gemessen. Die Messung wird insgesamt zehnmal durchgeführt, um ein genaueres Ergebnis zu erhalten. \\
\\
Um den Brechungsindex von Luft zu messen, wird eine Gaszelle in einen Strahl eingebaut, 
Der Doppelglashalter wird nicht ausgebaut. In die zuvor evakuierte Gaszelle wird nun Luft eingeleitet. Dann wird die Anzahl der Interferenzmaxima in 
Abhängigkeit vom Druck gemessen. Dazu wird er Druck in $50\,\mathrm{mbar}$-Schritten erhöht. Es wird außerdem die Raumtemperatur notiert. Die Messung des Brechungsindex 
von Luft wird zur besseren Genauigkeit dreimal wiederholt.