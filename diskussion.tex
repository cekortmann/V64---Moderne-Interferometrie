\section{Diskussion}
\label{sec:Diskussion}

Der Verlauf der Messwerte des Kontrasts in \autoref{fig:kontrast} liegen sehr nah an der Theoriekurve. Außerdem konnten sehr große Maxima 
des Kontrasts bei $K=0.943$ bei $\phi=50°$ und bei $K=0.952$ bei $\phi = 135°$ gemessen werden. Theoretisch wird das Maximum bei $45°$ 
erwartet. Die Abweichung dazu liegt in unserer Messung bei $11.11\%$. Diese lässt sich daraus erklären, dass zum einen nur im Abstand von 
$5°$ gemessen wurde und es so möglich ist, dass es ein Maximum gibt, dass dichter an dem theoretischen Wert liegt. Außerdem lässt sich die 
Abweichung dadurch erklären, dass das Interferometer nicht perfekt justiert war. Bereits kleine Erschütterungen ließen die Spiegel verrücken.