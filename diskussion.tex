\section{Diskussion}
\label{sec:Diskussion}

Der Verlauf der Messwerte des Kontrasts in \autoref{fig:kontrast} liegen sehr nah an der Theoriekurve. Außerdem konnten sehr große Maxima 
des Kontrasts bei $K=0.943$ bei $\phi=50°$ und bei $K=0.952$ bei $\phi = 135°$ gemessen werden. Theoretisch wird das Maximum bei $45°$ 
erwartet. Die Abweichung dazu liegt in unserer Messung bei $11.11\%$. Diese lässt sich daraus erklären, dass zum einen nur im Abstand von 
$5°$ gemessen wurde und es so möglich ist, dass es ein Maximum gibt, dass dichter an dem theoretischen Wert liegt. Außerdem lässt sich die 
Abweichung dadurch erklären, dass das Interferometer nicht perfekt justiert war. Bereits kleine Erschütterungen ließen die Spiegel verrücken.

Der Brechungsindex von Glas liegt bei circa 1.5. Dieser variiert je nach Glas.
 Wenn dies mit unserer Messung $n=1.556$ verglichen wird, kann festgestellt werden, dass es keine Abweichung gibt. Dies zeigt, dass die Messung 
sehr genau durchgeführt wurde. Die Abdeckung über dem Interferometer konnte Fluktuationen abschirmen.

Der Brechungsindex von Luft liegt bei 20°C bei $n=1.00029$. Bei 15°C liegt dieser bei 1.00028, sodass sich erkennen lässt, dass der Brechungsindex 
mit der Temperatur minimal ansteigt. Dennoch ergibt sich eine Abweichung zu unserem gemessenen Wert von $4.1\%$. Diese lässt sich damit erklären, 
dass die Zusammensetzung von Luft nicht an jedem Ort gleich ist und für den Theoriewert ein Durchschnittswert angenommen wird. Außerdem 
ließ sich die Anzahl der Maxima nicht so gut in der Abhängigkeit vom Druck ablesen. Auch Fluktuationen innerhalb des Interferometers konnten nicht 
gänzlich ausgeschlossen werden.